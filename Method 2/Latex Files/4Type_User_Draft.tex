\documentclass[fleqn]{article}
\usepackage[pdftex]{graphicx}
\DeclareGraphicsExtensions{.eps,.pdf,.jpeg,.png}

\usepackage[cmex10]{amsmath}
\usepackage[papersize={8.5in,11in}, left=0.68in, right=0.56in, top=0.8in, bottom=0.85in]{geometry}
\usepackage{amssymb}
%\usepackage{amsmath}
\usepackage{mathrsfs}
\usepackage{multirow}
%---------------------------------------------------------------------
\makeatletter
\renewcommand{\fnum@figure}{Fig. \thefigure}
\makeatother
\renewcommand{\baselinestretch}{1.2}

\newcommand{\argmax}{\operatornamewithlimits{argmax}}

\newcommand{\ignore}[1]{}


\usepackage[english]{babel}
%---------------------------------------------------------------------

\hyphenation{op-tical net-works semi-conduc-tor}


\begin{document}
\title{Fast Node Cardinality Estimation for 4 types of Nodes}
%\title{Design of a Cognitive MAC Protocol for Heterogeneous M2M Networks using Quickly Estimated Users}

\maketitle{}

In this table we specify 4 types and their corresponding symbols. 
\begin{center}
  \begin{tabular}{| c | c |}
    \hline
    Types & Symbols \\ \hline
    1 & $\alpha$ $0$ \\ \hline
    2 & $\alpha$ $\alpha$ \\ \hline
    3 & $0$ $\beta$ \\ \hline
    4 & $\beta$ $\beta$ \\ \hline
    
  \end{tabular}
\end{center}

2 slots are required in 1st phase. In each slot there are 4 possibilities (0 - Empty slot, $\alpha$, $\beta$, and C – Collision, where $\alpha$ and $\beta$ are two different symbols). Total of possible cases are ($4^2$ = 16).At least one collision cases are considered, since zero collision cases do not need
additional phase.
Exactly One collision cases are $\binom{2}{1}*3 = 6$. 
Here 0 implies empty slot and `Not Sure' outcome means additional slots are
required to identify the presence of these types of nodes.

\begin {table} [h]
\centering
\begin{tabular}{|c|c|c|c|} 
\hline
%\cline{1-5}
\multicolumn{2}{|c|}{Outcome in Block $i$} & \multicolumn{2}{c|}{Types} \\ \hline
Slot 1               & Slot 2              & Sure    & Not Sure         \\ \hline
C                    & 0                   & 1        & -                \\ \hline
C                    & $\alpha$            & 1,2     & -                \\ \hline
C                    & $\beta$             & 1       & One of \{3,4\}   \\ \hline
0                    & C                   & 3       & -                \\ \hline
$\alpha$             & C                   & 3       & One of (1,2)     \\ \hline
$\beta$              & C                   & 3,4     & -                \\ \hline
\end{tabular}
\caption{Exactly one collision case. \#, $C$ and $-$ denote ``Invalid Case'', ``Collision'' and  ``Nil'' respectively.}
\label{Tab_OneC1}
\end{table}


\end{document}
